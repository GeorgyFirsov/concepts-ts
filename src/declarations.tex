
%%
%% Declarations
%%
\setcounter{chapter}{9}
\rSec0[dcl.dcl]{Declarations}

Add \grammarterm{concept-definition} to the list of \grammarterm{declaration}s.

\begin{quote}
\begin{bnf}
\nontermdef{declaration}\br
  block-declaration\br
  nodeclspec-function-declaration\br
  function-definition\br
  template-declaration\br
  deduction-guide\br
  explicit-instantiation\br
  explicit-specialization\br
  linkage-specification\br
  namespace-definition\br
  \added{concept-definition}\br
  empty-declaration\br
  attribute-declaration
\end{bnf}
\end{quote}

Add the following paragraph at the end of this section.

\setcounter{Paras}{12}
\begin{quote}
\pnum
A \grammarterm{concept-definition} shall only appear as a part of 
\grammarterm{template-declaration}.
\end{quote}

%%
%% Specifiers
%%
\rSec1[dcl.spec]{Specifiers}


%%
%% Type specifiers
%%
\setcounter{subsection}{6}
\rSec2[dcl.type]{Type specifiers}

%%
%% Auto specifier
%%
\setcounter{subsubsection}{3}
\rSec3[dcl.spec.auto]{\tcode{auto} specifier}

Extend this section to allow \tcode{auto} to appear in multiple positions
within a single \grammarterm{type-specifier}.

Modify paragraph 1.

\begin{quote}
\pnum
The \tcode{auto} and \tcode{decltype(auto)} \grammarterm{type-specifier}{s}
are used to
designate a placeholder type that will be replaced later by deduction
from an initializer. The \tcode{auto}
\grammarterm{type-specifier} is also used to
introduce a function type having a \grammarterm{trailing-return-type} 
or to signify that a lambda is a generic lambda 
(\cxxref{expr.prim.lambda.closure}).
The \tcode{auto} \grammarterm{type-specifier} is also used to introduce a
structured binding declaration (\cxxref{dcl.struct.bind}).
\added{\enternote
A \grammarterm{nested-name-specifier} can also include placeholders (\ref{expr.prim}).
Replacements for those placeholders are determined according to the rules
in this section.
\exitnote}
\end{quote}

Modify paragraph 2 to allow multiple placeholder types.

\begin{quote}
\pnum
\removed{The placeholder type}\added{Placeholder types} can appear with a function 
declarator in the \grammarterm{decl-specifier-seq}, \grammarterm{type-specifier-seq},
\grammarterm{conversion-function-id}, or \grammarterm{trailing-return-type}, 
in any context where such a declarator is valid. 
% 
If the function declarator includes a \grammarterm{trailing-return-type} 
(\ref{dcl.fct}), that \grammarterm{trailing-return-type} specifies the declared 
return type of the function. Otherwise, the function declarator shall declare
a function.
%
If the declared return type of the function contains a placeholder, the 
return type of the function is deduced from non-discarded return statements,
if any, in the body of the function (\cxxref{stmt.if}).
\end{quote}

Modify paragraph 3 to allow the use of \tcode{auto} within the 
parameter type of a lambda or function.

\begin{quote}
\pnum
If the \tcode{auto} \grammarterm{type-specifier} 
appears \removed{as one of the \grammarterm{decl-specifier}{}s in the 
\grammarterm{decl-specifier-seq} of a \grammarterm{parameter-declaration}} 
\added{in a parameter type} of a \grammarterm{lambda-expression}, the lambda 
is a generic lambda
(\ref{expr.prim.lambda}).
%
\enterexample
\begin{codeblock}
auto glambda = [](int i, auto a) { return i; }; // OK: a generic lambda
\end{codeblock}
\exitexample
\end{quote}


Modify paragraph 4 in the \Cpp Standard (paragraph 7, here) to allow multiple 
placeholders within a variable declaration.

\begin{quote}
\pnum
The type of a variable declared using 
\added{a placeholder type} 
\removed{\tcode{auto} or \tcode{decltype(auto)}} 
is deduced from its initializer.
% 
This use is allowed in an initializing declaration (\cxxref{dcl.init}) of 
a variable.
%
\removed{\tcode{auto} or \tcode{decltype(auto)} shall appear as one
of the \grammarterm{decl-specifier}{}s in the \grammarterm{decl-specifier-seq} }
%
\added{
A placeholder type can appear anywhere in the declared type of the variable, but 
\tcode{decltype(auto)} shall appear only as one of the
\grammarterm{decl-specifier}{}s of the \grammarterm{decl-specifier-seq}.
}
\removed{and the}\added{The} \grammarterm{decl-specifier-seq} \added{of such a
variable} shall be followed by one or more \grammarterm{init-declarator}{}s,
each of which shall be followed by a non-empty initializer.
%
In an initializer of the form
\begin{codeblock}
( expression-list )
\end{codeblock}
the \grammarterm{expression-list} shall be a single 
\grammarterm{assignment-expression}.
% 
\enterexample
\begin{codeblock}
auto x = 5;                // OK: \tcode{x} has type int
const auto *v = &x, u = 6; // OK: \tcode{v} has type \tcode{const int*}, \tcode{u} has type \tcode{const int}
static auto y = 0.0;       // OK: \tcode{y} has type \tcode{double}
auto int r;                // error: \tcode{auto} is not a storage-class-specifier
auto f() -> int;           // OK: \tcode{f} returns \tcode{int}
auto g() { return 0.0; }   // OK: \tcode{g} returns \tcode{double}
auto h();                  // OK: \tcode{h}'s return type will be deduced when it is defined
\end{codeblock}
\exitexample
\end{quote}

Add the following declarations to the example in the previous paragraph.

\begin{quote}
\begin{addedblock}
\begin{codeblock}
struct N {
  template<typename T> struct Wrap;
  template<typename T> static Wrap<T> make_wrap(T);
};
template<typename T, typename U> struct Pair;
template<typename T, typename U> Pair<T, U> make_pair(T, U);
template<int N> struct Size { void f(int) { }  };

void (auto::* p1)(auto) = &Size<0>::f;   // OK: \tcode{p1} has type \tcode{void(Size<0>::*)(int)}
Pair<auto, auto> p2 = make_pair(0, 'a'); // OK: \tcode{p2} has type \tcode{Pair<int, char>}
N::Wrap<auto> a = N::make_wrap(0.0);     // OK: \tcode{a} has type \tcode{Wrap<double>}
auto::Wrap<int> x = N::make_wrap(0);     // error: failed to deduce value for \tcode{auto}
Size<sizeof(auto)> y = Size<0>{};        // error: failed to deduce value for \tcode{auto}
\end{codeblock}
\end{addedblock}
\end{quote}


Modify paragraph 7 in the \Cpp standard (paragraph 9, here) to read:

\begin{quote}
\pnum
If the \grammarterm{init-declarator-list} contains more than one 
\grammarterm{init-declarator}, they shall all form declarations of variables.
The type of each declared variable is determined is determined by placeholder 
type deduction (\ref{dcl.spec.auto.deduct}), and if 
the type that replaces \removed{the placeholder type} 
\added{the declared variable type or return type}
is not the same in each deduction, the program is ill-formed.
\end{quote}

Add add the following examples to that paragraph.

\begin{quote}
\enterexample
\begin{addedblock}
\begin{codeblock}
Pair<auto, auto> p1 = make_pair(0, 0), 
                 *p2 = &p1;              // OK: replacement type is \tcode{Pair<int, int>}
Pair<auto, auto> p3 = make_pair(0, 'a'), 
                 p4 = make_pair('a', 0); // error: different replacement types
\end{codeblock}
\end{addedblock}
\exitexample
\end{quote}


%%%
%%% Placeholder type deduction
%%%
\rSec4[dcl.spec.auto.deduct]{Placeholder type deduction}


\begin{quote}
\pnum
\defn{Placeholder type deduction} is the process by which a type containing
\removed{a placeholder type} \added{placeholder types} is replaced by
\removed{a deduced type} \added{deduced types)}.

\pnum
A type \tcode{T} containing a
\removed{placeholder type}
\added{placeholder types},
and a corresponding initializer \tcode{e},
are determined as follows:

\begin{itemize}
\item
for a non-discarded \tcode{return} statement that occurs
in a function declared with a return type
that contains 
\removed{a placeholder type}
\added{placeholder types},
\tcode{T} is the declared return type
and \tcode{e} is the operand of the \tcode{return} statement.
If the \tcode{return} statement
has no operand,
then \tcode{e} is \tcode{void()};
\item
for a variable declared with a type
that contains 
\removed{a placeholder type}
\added{placeholder types},
\tcode{T} is the declared type of the variable
and \tcode{e} is the initializer.
If the initialization is direct-list-initialization,
the initializer shall be a \grammarterm{braced-init-list}
containing only a single \grammarterm{assignment-expression}
and \tcode{e} is the \grammarterm{assignment-expression};
\item
for a non-type template parameter declared with a type
that contains 
\removed{a placeholder type}
\added{placeholder types},
\tcode{T} is the declared type of the non-type template parameter
and \tcode{e} is the corresponding template argument.
\end{itemize}
% 
In the case of a \tcode{return} statement with no operand
or with an operand of type \tcode{void},
\tcode{T} shall be either
\tcode{decltype(auto)} or \cv{}~\tcode{auto}.

\pnum
If the placeholder is the \tcode{auto} \grammarterm{type-specifier}, 
the deduced type
$\mathtt{T}'$ replacing \tcode{T} is determined using the rules for template 
argument deduction.
% 
\removed{Obtain \tcode{P} from
\tcode{T} by replacing the occurrences of \tcode{auto} with either a new
invented type template parameter \tcode{U} or,
if the initialization is copy-list-initialization, with
\tcode{std::initializer_list<U>}.}
%
\begin{addedblock}
Otherwise, obtain a type \tcode{P} from \tcode{T} as follows:
\begin{itemize}
% FIXME: This doesn't do the right thing at all.
\item if the initialization is a copy-list-initialization
and a placeholder type is a \grammarterm{decl-specifier} of the 
\grammarterm{decl-specifier-seq} of the variable declaration, replace that 
occurrence of the placeholder type with \tcode{std::initializer_list<U>}
where \tcode{U} is an invented type template parameter;

\item otherwise, replace each occurrence of a placeholder type in the
variable or return type with a new invented type template parameter
according to the rules for inventing template parameters
for placeholders in \ref{dcl.fct}. 
\end{itemize}
\end{addedblock}
%
Deduce a type for \added{each} \removed{\tcode{U}}
\added{invented type template parameter} using the rules
of template argument deduction from a function 
call~(\cxxref{temp.deduct.call}),
where \tcode{P} is a
function template parameter type and
the corresponding argument is \tcode{e}.
% 
If the deduction fails, the declaration is ill-formed.
%
Otherwise, $\mathtt{T}'$ is obtained by
substituting the deduced \removed{\tcode{U}}
\added{types for the invented type template parameters}
into \tcode{P}.

\enterexample
\begin{codeblock}
auto x1 = { 1, 2 };             // \tcode{decltype(x1)} is \tcode{std::initializer_list<int>}
auto x2 = { 1, 2.0 };           // error: cannot deduce element type
auto x3{ 1, 2 };                // error: not a single element
auto x4 = { 3 };                // \tcode{decltype(x4)} is \tcode{std::initializer_list<int>}
auto x5{ 3 };                   // \tcode{decltype(x5)} is \tcode{int}
\end{codeblock}
\exitexample

\enterexample
\begin{codeblock}
const auto &i = expr;
\end{codeblock}
The type of \tcode{i} is the deduced type of the parameter \tcode{u} in
the call \tcode{f(expr)} of the following invented function template:
\begin{codeblock}
template <class U> void f(const U& u);
\end{codeblock}
\exitexample
\end{quote}

Add the following to the first example in paragraph 4.

\begin{quote}
\begin{addedblock}
\enterexample
\begin{codeblock}
template<typename T> struct Vec { };
template<typename T> Vec<T> make_vec(std::initializer_list<T>) { return Vec<T>{}; }

template<typename... Ts> struct Tuple { };
template<typename... Ts> auto make_tup(Ts... args) { return Tuple<Ts...>{}; }

auto& x3 = *x1.begin();               // OK: \tcode{decltype(x3)} is \tcode{int\&}
const auto* p = &x3;                  // OK: \tcode{decltype(p)} is \tcode{const int*}
Vec<auto> v1 = make_vec({1, 2, 3});   // OK: \tcode{decltype(v1)} is \tcode{Vec<int>}
Vec<auto> v2 = {1, 2, 3};             // error: type deduction fails
Tuple<auto...> v3 = make_tup(0, 'a'); // OK: \tcode{decltype(v3)} is \tcode{Tuple<int, char>}
\end{codeblock}
\exitexample
\end{addedblock}
\end{quote}


Add the following after the second example in paragraph 4.

\begin{quote}
\begin{addedblock}
\enterexample
\begin{codeblock}
template<typename F, typename S> struct Pair;
template<typename T, typename U> Pair<T, U> make_pair(T, U);

struct S { void mfn(bool); } s;
int fn(char, double);

Pair<auto (*)(auto, auto), auto (auto::*)(auto)> p = make_pair(fn, &S::mfn);
\end{codeblock}
The declared type of \tcode{p} is the deduced type of the parameter 
\tcode{x} in the call of \tcode{g(make_pair(fn, \&S::mfn))} of the following 
invented function template:
\begin{codeblock}
template<class T1, class T2, class T3, class T4, class T5, class T6>
void g(Pair< T1(*)(T2, T3), T4 (T5::*)(T6)> x);
\end{codeblock}
\exitexample

\end{addedblock}
\end{quote}
