
%%
%% Declarators
%%
\rSec0[dcl.decl]{Declarators}

In paragraph 4, Modify the grammar of \grammarterm{declarator}{}s to
allow the specification of constraints on function declarations.

\begin{quote}
\setcounter{Paras}{3}
\pnum
Declarators have the syntax

\begin{bnf}
\nontermdef{declarator}\br
    ptr-declarator\br
    noptr-declarator parameters-and-qualifiers trailing-return-type
\end{bnf}

\begin{bnf}
\nontermdef{ptr-declarator}\br
    noptr-declarator\br
    ptr-operator ptr-declarator
\end{bnf}

\begin{bnf}
\nontermdef{noptr-declarator}\br
    declarator-id attribute-specifier-seq\opt\br
    noptr-declarator parameters-and-qualifiers\br
    noptr-declarator \terminal{[} constant-expression\opt{} \terminal{]} attribute-specifier-seq\opt\br
    \terminal{(} ptr-declarator \terminal{)}
\end{bnf}

\begin{bnf}
\nontermdef{parameters-and-qualifiers}\br
  \terminal{(} parameter-declaration-clause \terminal{)} cv-qualifier-seq\opt\br
    \hspace*{\bnfindentinc}ref-qualifier\opt exception-specification\opt attribute-specifier-seq\opt \added{requires-clause}\opt
\end{bnf}

\begin{bnf}
\nontermdef{trailing-return-type}\br
    \terminal{->} type-id
\end{bnf}

\begin{bnf}
\nontermdef{ptr-operator}\br
    \terminal{*} attribute-specifier-seq\opt cv-qualifier-seq\opt\br
    \terminal{\&} attribute-specifier-seq\opt\br
    \terminal{\&\&} attribute-specifier-seq\opt\br
    nested-name-specifier \terminal{*} attribute-specifier-seq\opt cv-qualifier-seq\opt
\end{bnf}

\begin{bnf}
\nontermdef{cv-qualifier-seq}\br
    cv-qualifier cv-qualifier-seq\opt
\end{bnf}

\begin{bnf}
\nontermdef{cv-qualifier}\br
    \terminal{const}\br
    \terminal{volatile}
\end{bnf}

\begin{bnf}
\nontermdef{ref-qualifier}\br
    \terminal{\&}\br
    \terminal{\&\&}
\end{bnf}

\begin{bnf}
\nontermdef{declarator-id}\br
    \terminal{...}\opt id-expression
\end{bnf}
\end{quote}

Add the following paragraph at the end of this section.

\begin{quote}
\begin{addedblock}
\pnum
The optional \grammarterm{requires-clause} (\ref{temp.constr.decl}) in a 
\grammarterm{declarator} shall be present only when the declarator declares a 
function (\ref{dcl.fct}), and that \grammarterm{requires-clause} shall not 
precede a \grammarterm{trailing-return-type}. 
% 
When present in a \grammarterm{declarator}, the \grammarterm{requires-clause} 
is called the \defn{trailing \grammarterm{requires-clause}{}}.
% 
\enterexample
\begin{codeblock}
void f1(int a) requires true;         // OK
auto f2(int a) -> bool requires true; // OK
auto f3(int a) requires true -> bool; // error: requires-clause precedes trailing-return-type
void (*pf)() requires true;           // error: constraint on a variable
void g(int (*)() requires true);      // error: constraint on a parameter-declaration
  
auto* p = new void(*)(char) requires true; // error: not a function declaration
\end{codeblock}
\exitexample
\end{addedblock}
\end{quote}


%%
%% Meaning of declarators
%%
\setcounter{section}{2}
\rSec1[dcl.meaning]{Meaning of declarators}


%%
%% Functions
%%
\setcounter{subsection}{4}
\rSec2[dcl.fct]{Functions}

Modify the matching condition in paragraph 1 to accept a 
\grammarterm{requires-clause}.
      
\begin{quote}
\pnum\vspace{-\the\baselineskip} % Adjust to hide the blank line.

\begin{bnf}
\terminal{D1} \terminal{(} parameter-declaration-clause \terminal{)} cv-qualifier-seq\opt\br
  \hspace*{\bnfindentinc}ref-qualifier\opt exception-specification\opt attribute-specifier-seq\opt \added{requires-clause\opt}
\end{bnf}
\end{quote}

Modify the matching condition in paragraph 2 to accept a 
\grammarterm{requires-clause}.

\begin{quote}
\pnum\vspace{-\the\baselineskip} % Adjust to hide the blank line.

\begin{bnf}
\terminal{D1} \terminal{(} parameter-declaration-clause \terminal{)} cv-qualifier-seq\opt\br 
  \hspace*{\bnfindentinc}ref-qualifier\opt exception-specification\opt attribute-specifier-seq\opt\br
  \hspace*{\bnfindentinc}trailing-return-type \added{requires-clause\opt}
\end{bnf}
\end{quote}


Modify the first part of paragraph 5. The unchanged remainder of the paragraph
is omitted.

\begin{quote}
\setcounter{Paras}{4}
\pnum
A single name can be used for several different functions in a single 
scope; this is function overloading (Clause~\ref{over}). 
%
All declarations for a function shall agree exactly in \removed{both} 
the return type\added{,} \removed{and} the parameter-type-list\added{, and 
associated constraints, if any (\ref{temp.constr.decl})}.
\end{quote}

Modify paragraph 8 to exclude constraints from the type of a function.
Note that the change occurs in the sentence following the example
in the \Cpp Standard.

\begin{quote}
\setcounter{Paras}{7}
\pnum
The return type, the parameter-type-list, the \grammarterm{ref-qualifier}, 
the \grammarterm{cv-qualifier-seq}, and the exception specification, but not 
the default arguments (\cxxref{dcl.fct.default})
\added{or associated constraints (\ref{temp.constr.decl})}
are part of the function type.
\end{quote}
