
%%
%% Template constraints
%%
\setcounter{section}{9}
\rSec1[temp.constr]{Template constraints}

Add this section after \cxxref{temp.deduct.guide} in the \Cpp standard.

\begin{quote}
\begin{addedblock}

\pnum
\enternote
This section defines the meaning of constraints on template arguments.
% 
The abstract syntax and satisfaction rules are defined
in \ref{temp.constr.constr}. 
% 
Constraints are associated with declarations in \ref{temp.constr.decl}.
% 
Declarations are partially ordered by their associated constraints 
(\ref{temp.constr.order}).
\exitnote


%%
%% Constraints
%%
\rSec2[temp.constr.constr]{Constraints}

\pnum
A \defn{constraint} is a sequence of logical operations and 
operands that specifies requirements on template arguments.
\enternote The operands of a logical operation are constraints. \exitnote
% 
There are several different kinds of constraints:
\begin{itemize}
\item conjunctions (\ref{temp.constr.op}),
\item disjunctions (\ref{temp.constr.op}), and
\item atomic constraints (\ref{temp.constr.atomic})
\end{itemize}

\pnum
In order for a constrained template to be instantiated (\ref{temp.spec}), its 
associated constraints shall be \defn{satisfied} (\ref{temp.constr.decl}).
% 
\enternote
The satisfaction of constraints on class templates, alias templates, 
and variable templates is required when referring to a template specialization 
(\ref{temp.names}). The satisfaction of constraints on functions and
function templates is required during overload resolution 
(\ref{over.match.viable}).
\exitnote
% 
The rules for determining the satisfaction of different kinds of 
constraints are defined in the following subsections.


%%
%% Logical operations
%%
\rSec3[temp.constr.op]{Logical operations}

\pnum
There are two binary logical operations on constraints: conjunction
and disjunction.
% 
\enternote 
These logical operations have no corresponding \Cpp syntax.
For the purpose of exposition, conjunction is spelled
using the symbol $\land$ and disjunction is spelled using the 
symbol $\lor$. 
% 
The operands of these operations are called the left 
and right operands. In the constraint $A \land B$,
$A$ is the left operand, and $B$ is the right operand.
% 
Grouping of constraints is shown using parentheses.
\exitnote

\pnum
A \defn{conjunction} is a constraint taking two 
operands. 
% 
To determine if a conjunction is satisfied, check the satisfaction of
the left operand. If that is not satisfied, the conjunction is not
satisfied. Otherwise, the conjunction is satisfied if and only if the right
operand is satisfied.

\pnum
A \defn{disjunction} is a constraint taking two 
operands. 
% 
To determine if a disjunction is satisfied, check the satisfaction of
the left operand. If that is satisfied, the disjunction is
satisfied. Otherwise, the disjunction is satisfied if and only if the right
operand is satisfied.

\pnum
\enterexample
\begin{codeblock}
template<typename T>
  constexpr bool get_value() { return T::value; }

template<typename T>
  requires sizeof(T) > 1 && get_value<T>()
    void f(T);   // has associated constraint \tcode{sizeof(T) > 1 $\land$ get_value<T>()}

void f(int);

f('a'); // OK: calls \tcode{f(int)}
\end{codeblock}
In the satisfaction of the associated constraints (\ref{temp.constr.decl}) 
of \tcode{f}, the constraint \tcode{sizeof(char) > 1} is not satisfied; 
the second operand is not checked for satisfaction.
\exitexample


%%
%% Atomic constraints
%%
\rSec3[temp.constr.atomic]{Atomic constraints}

\pnum
An \defn{atomic constraint} is a constraint that evaluates a constant 
expression \tcode{E} (\cxxref{expr.const}) that is neither a logical
AND expression (\cxxref{expr.log.and}) nor a logical OR expression
(\cxxref{expr.log.or}), under a mapping from the template parameters
that appear within \tcode{E} to template arguments involving the
template parameters of the constrained entity (the \defn{parameter
mapping} of the atomic constraint).
Two atomic constraints are equivalent if they are formed from the same
\grammarterm{expression} and the parameter mappings are equivalent
according to the rules for expressions described in \ref{temp.over.link},
except that any \grammarterm{identifier}{s} referring to 
local parameters of \grammarterm{requires-expression}{s}
are equivalent if and only if the types of their corresponding
declarations are equivalent (\cxxref{temp.type}).
% 
\enternote
Atomic constraints are formed by constraint normalization (\ref{temp.constr.decl}).
\exitnote
% 
Determining if a constraint is satisfied entails the substitution 
of the parameter mapping and template arguments into that constraint.
% 
After substitution, \tcode{E} shall have type \tcode{bool}.
% 
The constraint is satisfied if and only if \tcode{E} evaluates to 
\tcode{true}.
% 
\enterexample
\begin{codeblock}
template<typename T> 
  concept C = sizeof(T) == 4 && !true; // requires predicate constraints
                                       // \tcode{sizeof(T) == 4} and \tcode{!true}

template<typename T>
  struct S {
    constexpr explicit operator bool() const { return true; }
  };

template<typename T>
  requires S<T>{}
    void f(T);

f(0); // error: constraints cannot be satisfied because the
      // expression \tcode{S<int>\{\}} does not have type \tcode{bool}
\end{codeblock}
No conversions are applied to atomic constraints.
\exitexample


%%
%% Constrained declarations
%%
\rSec2[temp.constr.decl]{Constrained declarations}

\pnum
A template declaration (Clause~\ref{temp}) or function declaration 
(\ref{dcl.fct}) can be constrained by the use of a 
\grammarterm{requires-clause}. 
% 
This allows the specification of constraints for that declaration as
an expression:

\begin{bnf}
\nontermdef{constraint-expression}\br
    logical-or-expression
\end{bnf}

\pnum
Constraints can also be associated with a declaration through the use of 
\grammarterm{constrained-parameter}{}s in a 
\grammarterm{template-parameter-list}.
% 
Each of these forms introduces additional \grammarterm{constraint-expression}{s} 
that are used to constrain the declaration.
% 
A template's \defn{associated constraints} are defined as a 
single \grammarterm{constraint-expression} derived from the
introduced \grammarterm{constraint-expression}{s} using the
following rules.

\begin{itemize}
\item If there are no introduced \grammarterm{constraint-expression}{s},
the declaration is unconstrained.

\item If there is a single introduced \grammarterm{constraint-expression},
that is the associated constraint.

\item Otherwise, the associated constraints are formed as a logical 
AND expression (\cxxref{expr.log.and}) whose operands are in the following order:
% 
\begin{itemize}
\item the \grammarterm{constraint-expression} introduced by each
      \grammarterm{constrained-parameter} (\ref{temp.param}) in the 
      declaration's \grammarterm{template-parameter-list}, in
      order of appearance, and

\item the \grammarterm{constraint-expression} introduced
      by a \grammarterm{requires-clause} following a 
      \grammarterm{template-parameter-list} (Clause~\ref{temp}), and

\item the \grammarterm{constraint-expression} of a trailing 
      \grammarterm{requires-clause} (Clause~\ref{dcl.decl}) 
      of a function declaration (\ref{dcl.fct}).
\end{itemize}
\end{itemize}
% 
The formation of the associated constraints for a template declaration
establishes the order in which constraints are instantiated when checking 
for satisfaction (\ref{temp.constr.constr}).
% 
\enternote
These constraints are checked during the instantiation of the declaration.
\exitnote
% 
\enterexample
\begin{codeblock}
template<typename T> concept C = true;

template<C T> void f1(T);
template<typename T> requires C<T> void f2(T);
template<typename T> void f3(T) requires C<T>;
\end{codeblock}
The associated constraints of \tcode{f1}, \tcode{f2}, and \tcode{f3}
are \tcode{C<T>}.
% 
\begin{codeblock}
template<typename T> concept C1 = true;
template<typename T> concept C2 = sizeof(T) > 0;

template<C1 T> void f6(T) requires C2<T>;
template<typename T> requires C1<T> && C2<T> void f7(T);
\end{codeblock}
The associated constraints of \tcode{f6} and \tcode{f7} are both
\tcode{C1<T> \&\& C2<T>}.
% 
\begin{codeblock}
template<C1 T> requires C2<T> void f8();
template<C2 T> requires C1<T> void f9();
\end{codeblock}
% 
The associated constraints of \tcode{f8} are
\tcode{C1<T> \&\& C2<T>}, and those \tcode{f9} are
\tcode{C2<T> \&\& C1<T>}.
\exitexample

\pnum
Determining the satisfaction of a declaration's associated constraints,
and the partial ordering of declarations by those constraints,
requires that they are first \defn{normalized}.
% 
Normalization transforms a \grammarterm{expression} into a sequence of 
conjunctions and disjunctions (\ref{temp.constr.op})
of atomic constraints (\ref{temp.constr.atomic}).
% 
The \defn{normal form} of an \grammarterm{expression} \tcode{X} is 
defined as follows:
% 
First, replace all \grammarterm{id-expression}{s} of the form 
\tcode{C<A$1$, A$2$, ..., A$N$>}, where \tcode{C} is a concept with the 
result of substituting the template arguments \tcode{A$1$, A$2$, ..., A$N$} 
into the \grammarterm{constraint-expression} of \tcode{C}.
% 
If that substitution fails, the program is ill-formed.
% 
Second, transform the expression \tcode{E} into a constraint as follows:
\begin{itemize}
\item the normal form of an expression \tcode{(E)} is the normal form
of \tcode{E};

\item The normal form of an expression \tcode{E1 || E2} is the  
disjunction (\ref{temp.constr.op}) of the normal forms of \tcode{E1} 
and \tcode{E2}.

\item The normal form of an expression \tcode{E1 \&\& E2} is the  
conjunction of the normal forms of \tcode{E1} 
and \tcode{E2}.

\item The normal form of an \grammarterm{id-expression} of the form
\tcode{C<A$1$, A$2$, ..., A$N$>}, where \tcode{C} names the concept
\tcode{D} (\ref{dcl.spec.concept}),
is the normal form of the initializer of \tcode{D},
after substituting \tcode{A$1$, A$2$, ..., A$N$} for
\tcode{D}{'s} respective template parameters in the
parameter mappings in each atomic constraint.
If any such substitution fails, the program is ill-formed.

\item The normal form of a \grammarterm{requires-expression} \tcode{E} is
the conjunction of:
\begin{itemize}
\item the atomic constraint whose expression is \tcode{E}, and
\item for each \grammarterm{nested-requirement} in \tcode{E}, the
normal form of the \grammarterm{constraint-expression}.
\end{itemize}

\item otherwise, the normal form of \tcode{E} is the atomic constraint
whose expression is \tcode{E} and whose parameter mapping is the
identity mapping.
\end{itemize}
% 
A declaration's \defn{normalized constraints} are those
yielded by normalizing its associated constraints.
% 
\enterexample
\begin{codeblock}
template<typename T> concept C1 = sizeof(T) == 1;
template<typename T> concept C2 = C1<T>() && 1 == 2;
template<typename T> concept C3 = requires { typename T::type; };
template<typename T> concept C4 = requires (T x) { ++x; }

template<C2 U> void f1(U);                            // \#1
template<C3 U> void f2(U);                            // \#2
template<C4 U> void f3(U);                            // \#3
template<typename T> requires (bool)3 + 4 void f4(T); // error: invalid constraints (\#4)
\end{codeblock}
The normalized associated constraints of \#1 are 
$\tcode{sizeof(T) == 1}$ (with mapping \tcode{T}$\mapsto$\tcode{U}) $\land \tcode{1 == 2}$,
% 
those of \#2 are
$\tcode{requires \{ typename T::type; \}}$ (with mapping \tcode{T}$\mapsto$\tcode{U}),
%
those of \#3 are
$\tcode{requires (T x) \{ ++x; \}}$ (with mapping \tcode{T}$\mapsto$\tcode{U}).
% 
In \#4, the \grammarterm{constraint-expression} \tcode{(bool)3 + 4}
is not a valid predicate constraint because it does not have type \tcode{bool}.
\exitexample


\pnum
A declaration's associated constraints are satisfied by a set of template
arguments if and only if its normalized associated constraints are satisfied
by those arguments.


%%
%% Partial ordering by constraints
%%
\rSec2[temp.constr.order]{Partial ordering by constraints}

\pnum
A constraint $P$ is said to \defn{subsume} another constraint $Q$ 
if, informally, it can be determined that $P$ implies $Q$, up to 
the equivalence of atomic constraints in $P$ and $Q$
(\ref{temp.constr.atomic}).
% 
\enterexample
Subsumption does not determine if the atomic constraint 
\tcode{N >= 0} (\ref{temp.constr.atomic}) subsumes \tcode{N > 0} for some 
integral template argument \tcode{N}.
\exitexample

\pnum
In order to determine if a constraint $P$ subsumes a constraint
$Q$, transform $P$ into disjunctive normal form, 
and transform $Q$ into conjunctive normal form\footnote{
A constraint is in disjunctive normal form when it is a disjunction of
clauses where each clause is a conjunction of atomic constraints. 
% 
Similarly, a constraint is in conjunctive normal form when it is a conjunction 
of clauses where each clause is a disjunction of atomic constraints.
% 
\enterexample
Let $A$, $B$, and $C$ be atomic constraints.
% 
The constraint $A \land$ ($B \lor C$) is in 
conjunctive normal form.
% 
Its conjunctive clauses are $A$ and ($B \lor C$).
% 
The disjunctive normal form of the constraint
$A \land$ ($B \lor C$) 
is
($A \land B$) $\lor$ ($A \land C$).
% 
Its disjunctive clauses are ($A \land B$) and 
($A \land C$).
\exitexample
}.
% 
Then, $P$ subsumes $Q$ if and only if
\begin{itemize}
\item for every disjunctive clause $P_i$ in the disjunctive normal 
form of $P$, $P_i$ subsumes every conjunctive clause $Q_j$ 
in the conjuctive normal form of \tcode{Q}, where

\item a disjunctive clause $P_i$ subsumes a conjunctive clause
$Q_j$ if and only if each atomic constraint in $P_i$ subsumes 
any atomic constraint $Q_j$, where

\item an atomic constraint $A$ subsumes another atomic constraint
$B$ if and only if the $A$ and $B$ are equivalent using the
rules described in \ref{temp.constr.atomic}.
\end{itemize}
% 
\enterexample
Let $A$ and $B$ be atomic constraints (\ref{temp.constr.atomic}).
% 
The constraint $A \land B$ subsumes $A$, but $A$ does not subsume $A \land B$.
% 
The constraint $A$ subsumes $A \lor B$, but $A \lor B$ does not subsume $A$.
% 
Also note that every constraint subsumes itself.
\exitexample


\pnum
The subsumption relation defines a partial ordering on constraints. 
This partial ordering is used to determine
% 
\begin{itemize}
\item the best viable candidate of non-template functions
     (\ref{over.match.best}), 
\item the address of a non-template function
     (\ref{over.over}), 
\item the matching of template template arguments
     (\ref{temp.arg.template}), 
\item the partial ordering of class template specializations
     (\ref{temp.class.order}), and
\item the partial ordering of function templates
     (\ref{temp.func.order}).
\end{itemize}

%%% FIXME: We need to substitute the deductions from partial ordering
%%% into the constraints before comparing them, otherwise they will be
%%% referring to unrelated template parameters.
\pnum
When two declarations \tcode{D1} and \tcode{D2} are
partially ordered by their normalized constraints (\ref{temp.constr.decl}), 
\tcode{D1} is \defn{at least as constrained} as \tcode{D2} if
% 
\begin{itemize}
\item \tcode{D1} and \tcode{D2} are both constrained declarations and 
\tcode{D1}'s normalized constraints subsume those of \tcode{D2}; or

\item \tcode{D2} is unconstrained.
\end{itemize}

\pnum
A declaration \tcode{D1} is \defn{more constrained}
than another declaration \tcode{D2} when \tcode{D1} is at least as
constrained as \tcode{D2}, and \tcode{D2} is not at least as
constrained as \tcode{D1}.

\enterexample
\begin{codeblock}
template<typename T> concept C1 = requires(T t) { --t; };
template<typename T> concept C2 = C1<T> && requires(T t) { *t; };

template<C1 T> void f(T);       // \#1
template<C2 T> void f(T);       // \#2
template<typename T> void g(T); // \#3
template<C1 T> void g(T);       // \#4

f(0);       // selects \#1
f((int*)0); // selects \#2
g(true);    // selects \#3 because \tcode{C1<bool>} is not satisfied
g(0);       // selects \#4
\end{codeblock}
\exitexample

\end{addedblock}
\end{quote}
