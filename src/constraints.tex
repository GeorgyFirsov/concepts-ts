
%%
%% Template constraints
%%
\setcounter{section}{9}
\rSec1[temp.constr]{Template constraints}

Add this section after \cxxref{temp.deduct.guide} in the \Cpp standard.

\begin{quote}
\begin{addedblock}

\pnum
\enternote
This section defines the meaning of constraints on template arguments.
% 
The abstract syntax and satisfaction rules are defined
in \ref{temp.constr.constr}. 
% 
Constraints are associated with declarations in \ref{temp.constr.decl}.
% 
Declarations are partially ordered by their associated constraints 
(\ref{temp.constr.order}).
\exitnote


%%
%% Constraints
%%
\rSec2[temp.constr.constr]{Constraints}

\pnum
A \defn{constraint} is a sequence of logical operations and 
operands that specifies requirements on template arguments.
The operands of a logical operation are constraints.
% 
There are several different kinds of constraints:
\begin{itemize}
\item conjunctions (\ref{temp.constr.op}),
\item disjunctions (\ref{temp.constr.op}), and
\item atomic constraints (\ref{temp.constr.atomic})
\end{itemize}

\pnum
In order for a constrained template to be instantiated (\ref{temp.spec}), its 
associated constraints shall be \defn{satisfied} (\ref{temp.constr.decl}).
% 
\enternote
Forming a template specialization name (\ref{temp.names}) of a class template, 
variable template, or an alias template requires the satisfaction of its 
constraints.
Overload resolution (\ref{over.match.viable}) requires the satisfaction of
constraints on functions and function templates.
\exitnote
% 
The rules for determining the satisfaction of different kinds of 
constraints are defined in the following subsections.


%%
%% Logical operations
%%
\rSec3[temp.constr.op]{Logical operations}

\pnum
There are two binary logical operations on constraints: conjunction
and disjunction.
% 
\enternote 
These logical operations have no corresponding \Cpp syntax.
For the purpose of exposition, conjunction is spelled
using the symbol $\land$ and disjunction is spelled using the 
symbol $\lor$. 
% 
The operands of these operations are called the left 
and right operands. In the constraint $A \land B$,
$A$ is the left operand, and $B$ is the right operand.
\exitnote

\pnum
A \defn{conjunction} is a constraint taking two 
operands. 
% 
To determine if a conjunction is satisfied, the satisfaction of
the first operand is checked. If that is not satisfied, the conjunction is not
satisfied. Otherwise, the conjunction is satisfied if and only if the second
operand is satisfied.

\pnum
A \defn{disjunction} is a constraint taking two 
operands. 
% 
To determine if a disjunction is satisfied, the satisfaction of
the first operand is checked. If that is satisfied, the disjunction is
satisfied. Otherwise, the disjunction is satisfied if and only if the second
operand is satisfied.

\pnum
\enterexample
\begin{codeblock}
template<typename T>
  constexpr bool get_value() { return T::value; }

template<typename T>
  requires sizeof(T) > 1 && get_value<T>()
    void f(T);   // has associated constraint \tcode{sizeof(T) > 1 $\land$ get_value<T>()}

void f(int);

f('a'); // OK: calls \tcode{f(int)}
\end{codeblock}
In the satisfaction of the associated constraints (\ref{temp.constr.decl}) 
of \tcode{f}, the constraint \tcode{sizeof(char) > 1} is not satisfied; 
the second operand is not checked for satisfaction.
\exitexample


%%
%% Atomic constraints
%%
\rSec3[temp.constr.atomic]{Atomic constraints}

\pnum
An \defn{atomic constraint} is formed from
an expression \tcode{E}
and a mapping from the template parameters
that appear within \tcode{E} to
template arguments involving the
template parameters of the constrained entity,
called the \defn{parameter mapping} (\ref{temp.constr.decl}).
%
Two atomic constraints are \defn{identical} if they are formed from the same
\grammarterm{expression} and the parameter mappings are equivalent
according to the rules for expressions described in \ref{temp.over.link}.
% 
\enternote
Atomic constraints are formed by constraint normalization (\ref{temp.constr.normal}).
\tcode{E} is never a logical AND expression (\cxxref{expr.log.and})
nor a logical OR expression (\cxxref{expr.log.or}).
\exitnote
% 
Determining if a constraint is satisfied entails the substitution 
of the parameter mapping and template arguments into that constraint.
% 
After substitution, \tcode{E} shall have type \tcode{bool}.
% 
The constraint is satisfied if and only if evaluation of \tcode{E}
as a constant expression (\cxxref{expr.const}) succeeds and \tcode{E}
evaluates to \tcode{true}.
% 
\enterexample
\begin{codeblock}
template<typename T> 
  concept C = sizeof(T) == 4 && !true; // requires atomic constraints
                                       // \tcode{sizeof(T) == 4} and \tcode{!true}

template<typename T>
  struct S {
    constexpr operator bool() const { return true; }
  };

template<typename T>
  requires (S<T>{})
    void f(T);

void g() {
  f(0); // error: constraints cannot be satisfied because the
        // expression \tcode{S<int>\{\}} does not have type \tcode{bool};
        // no conversions are applied to atomic constraints.
}
\end{codeblock}
\exitexample


%%
%% Constrained declarations
%%
\rSec2[temp.constr.decl]{Constrained declarations}

\pnum
A template declaration (Clause~\ref{temp}) or function declaration 
(\ref{dcl.fct}) can be constrained by the use of a 
\grammarterm{requires-clause}. 
% 
This allows the specification of constraints for that declaration as
an expression:

\begin{bnf}
\nontermdef{constraint-expression}\br
    logical-or-expression
\end{bnf}

\pnum
Constraints can also be associated with a declaration through the use of 
\grammarterm{constrained-parameter}{}s in a 
\grammarterm{template-parameter-list}.
% 
Each of these forms introduces additional \grammarterm{constraint-expression}{s} 
that are used to constrain the declaration.

\pnum
A template's \defn{associated constraints} are defined as follows:
%
\begin{itemize}
\item If there are no introduced \grammarterm{constraint-expression}{s},
the declaration has no associated constraints.

\item If there is a single introduced \grammarterm{constraint-expression},
the associated constraints are the normal form (\ref{temp.constr.normal})
of that expression.

\item Otherwise, the associated constraints are the normal form of a logical 
AND expression (\cxxref{expr.log.and}) whose operands are in the 
following order:
% 
\begin{itemize}
\item the \grammarterm{constraint-expression} introduced by each
      \grammarterm{constrained-parameter} (\ref{temp.param}) in the 
      declaration's \grammarterm{template-parameter-list}, in
      order of appearance, and

\item the \grammarterm{constraint-expression} introduced
      by a \grammarterm{requires-clause} following a 
      \grammarterm{template-parameter-list} (Clause~\ref{temp}), and

\item the \grammarterm{constraint-expression} introduced by a trailing 
      \grammarterm{requires-clause} (Clause~\ref{dcl.decl}) 
      of a function declaration (\ref{dcl.fct}).
\end{itemize}
\end{itemize}
% 
The formation of the associated constraints
establishes the order in which constraints are instantiated when checking 
for satisfaction (\ref{temp.constr.constr}).
% 
\enterexample
\begin{codeblock}
template<typename T> concept C = true;

template<C T> void f1(T);
template<typename T> requires C<T> void f2(T);
template<typename T> void f3(T) requires C<T>;
\end{codeblock}
The functions \tcode{f1}, \tcode{f2}, and \tcode{f3} have the associated
constraint \tcode{C<T>}.
% 
\begin{codeblock}
template<typename T> concept C1 = true;
template<typename T> concept C2 = sizeof(T) > 0;

template<C1 T> void f4(T) requires C2<T>;
template<typename T> requires C1<T> && C2<T> void f5(T);
\end{codeblock}
The associated constraints of \tcode{f4} and \tcode{f5}
are \tcode{C1<T> $\land$ C2<T>}.
% 
\begin{codeblock}
template<C1 T> requires C2<T> void f6();
template<C2 T> requires C1<T> void f7();
\end{codeblock}
The associated constraints of 
\tcode{f6} are \tcode{C1<T> $\land$ C2<T>},
and those of 
\tcode{f7} are \tcode{C2<T> $\land$ C1<T>}.
\exitexample

\rSec2[temp.constr.normal]{Constraint normalization}

\pnum
Determining a declaration's
associated constraints (\ref{temp.constr.constr}), 
requires that their \grammarterm{constraint-expression}{s}
are \defn{normalized}.
% 
Normalization transforms a \grammarterm{constraint-expression} into a 
sequence of conjunctions and disjunctions (\ref{temp.constr.op})
of atomic constraints (\ref{temp.constr.atomic}).
% 
The \defn{normal form} of an \grammarterm{expression} \tcode{E} is 
defined as follows:
% 
\begin{itemize}
\item The normal form of an expression \tcode{(E)} is the normal form
of \tcode{E}.

\item The normal form of an expression \tcode{E1 || E2} is the  
disjunction (\ref{temp.constr.op}) of the normal forms of \tcode{E1} 
and \tcode{E2}.

\item The normal form of an expression \tcode{E1 \&\& E2} is the  
conjunction of the normal forms of \tcode{E1} 
and \tcode{E2}.

\item The normal form of an \grammarterm{id-expression} of the form
\tcode{C<A$1$, A$2$, ..., A$N$>}, where \tcode{C} names a concept,
is the normal form of the \grammarterm{constraint-expression} of \tcode{C},
after substituting \tcode{A$1$, A$2$, ..., A$N$} for
\tcode{C}{'s} respective template parameters in the
parameter mappings in each atomic constraint.
If any such substitution results in an invalid type or expression,
the program is ill-formed; no diagnostic is required.
\enterexample
\begin{codeblock}
template<typename T> concept A = T::value || true;
template<typename U> concept B = A<U*>;
template<typename V> concept C = B<V&>;
\end{codeblock}
Normalization of \tcode{B}'s \grammarterm{constraint-expression}
is valid and results in
\tcode{T::value} (with the mapping \tcode{T}$\mapsto$\tcode{U*})
$\land$
\tcode{true} (with an empty mapping),
despite the expression \tcode{T::value} being ill-formed
for a pointer type \tcode{T}.
Normalization of \tcode{C}'s \grammarterm{constraint-expression}
results in the program being ill-formed,
because it would form the invalid type \tcode{T\&*}
in the parameter mapping.
\exitexample

\item The normal form of any other expression \tcode{E}
is the atomic constraint
whose expression is \tcode{E} and whose parameter mapping is the
identity mapping.
\end{itemize}

\enterexample
\begin{codeblock}
template<typename T> concept C1 = sizeof(T) == 1;
template<typename T> concept C2 = C1<T>() && 1 == 2;
template<typename T> concept C3 = requires { typename T::type; };
template<typename T> concept C4 = requires (T x) { ++x; }

template<C2 U> void f1(U);                            // \#1
template<C3 U> void f2(U);                            // \#2
template<C4 U> void f3(U);                            // \#3
\end{codeblock}
The associated constraints of \#1 are 
$\tcode{sizeof(T) == 1}$ (with mapping \tcode{T}$\mapsto$\tcode{U}) $\land \tcode{1 == 2}$,
% 
those of \#2 are
$\tcode{requires \{ typename T::type; \}}$ (with mapping \tcode{T}$\mapsto$\tcode{U}),
%
those of \#3 are
$\tcode{requires (T x) \{ ++x; \}}$ (with mapping \tcode{T}$\mapsto$\tcode{U}).
\exitexample


%%
%% Partial ordering by constraints
%%
\rSec2[temp.constr.order]{Partial ordering by constraints}

\pnum
A constraint $P$ is said to subsume another constraint $Q$ 
if it can be determined that $P$ implies $Q$, up to 
the identity of atomic constraints in $P$ and $Q$
(\ref{temp.constr.atomic}), as described below.
% 
\enterexample
Subsumption does not determine if the atomic constraint 
\tcode{N >= 0} (\ref{temp.constr.atomic}) subsumes \tcode{N > 0} for some 
integral template argument \tcode{N}.
\exitexample

\pnum
In order to determine if a constraint $P$ \defn{subsumes} a constraint
$Q$, $P$ is transformed into disjunctive normal form, 
and $Q$ is transformed into conjunctive normal form\footnote{
A constraint is in disjunctive normal form when it is a disjunction of
clauses where each clause is a conjunction of atomic constraints. 
% 
Similarly, a constraint is in conjunctive normal form when it is a conjunction 
of clauses where each clause is a disjunction of atomic constraints.
% 
\enterexample
Let $A$, $B$, and $C$ be atomic constraints.
% 
The constraint $A \land$ ($B \lor C$) is in 
conjunctive normal form.
% 
Its conjunctive clauses are $A$ and ($B \lor C$).
% 
The disjunctive normal form of the constraint
$A \land$ ($B \lor C$) 
is
($A \land B$) $\lor$ ($A \land C$).
% 
Its disjunctive clauses are ($A \land B$) and 
($A \land C$).
\exitexample
}.
% 
Then, $P$ subsumes $Q$ if and only if
\begin{itemize}
\item for every disjunctive clause $P_i$ in the disjunctive normal 
form of $P$, $P_i$ subsumes every conjunctive clause $Q_j$ 
in the conjuctive normal form of $Q$, where

% \item a disjunctive clause $P_i$ subsumes a conjunctive clause
% $Q_j$ if and only if each atomic constraint in $P_i$ subsumes 
% any atomic constraint in $Q_j$, where

\item
a disjunctive clause $P_i$ subsumes a conjunctive clause $Q_j$ if and only 
if there exists an atomic constraint $P_{ia}$ in $P_i$ for which there exists 
an atomic constraint, $Q_{jb}$, in $Q_j$ such that $P_{ia}$ subsumes $Q_{jb}$.

\item an atomic constraint $A$ subsumes another atomic constraint
$B$ if and only if the $A$ and $B$ are identical using the
rules described in \ref{temp.constr.atomic}.
\end{itemize}
% 
\enterexample
Let $A$ and $B$ be atomic constraints (\ref{temp.constr.atomic}).
% 
The constraint $A \land B$ subsumes $A$, but $A$ does not subsume $A \land B$.
% 
The constraint $A$ subsumes $A \lor B$, but $A \lor B$ does not subsume $A$.
% 
Also note that every constraint subsumes itself.
\exitexample


\pnum
\enternote
The subsumption relation defines a partial ordering on constraints. 
This partial ordering is used to determine
% 
\begin{itemize}
\item the best viable candidate of non-template functions
     (\ref{over.match.best}), 
\item the address of a non-template function
     (\ref{over.over}), 
\item the matching of template template arguments
     (\ref{temp.arg.template}), 
\item the partial ordering of class template specializations
     (\ref{temp.class.order}), and
\item the partial ordering of function templates
     (\ref{temp.func.order}).
\end{itemize}
\exitnote

%%% FIXME: We need to substitute the deductions from partial ordering
%%% into the constraints before comparing them, otherwise they will be
%%% referring to unrelated template parameters.
\pnum
When two declarations \tcode{D1} and \tcode{D2} are
partially ordered by their associated constraints (\ref{temp.constr.decl}), 
\tcode{D1} is \defn{at least as constrained} as \tcode{D2} if
% 
\begin{itemize}
\item \tcode{D1} and \tcode{D2} are both constrained declarations and 
\tcode{D1}'s associated constraints subsume those of \tcode{D2}; or

\item \tcode{D2} has no associated constraints.
\end{itemize}

\pnum
A declaration \tcode{D1} is \defn{more constrained}
than another declaration \tcode{D2} when \tcode{D1} is at least as
constrained as \tcode{D2}, and \tcode{D2} is not at least as
constrained as \tcode{D1}.

\enterexample
\begin{codeblock}
template<typename T> concept C1 = requires(T t) { --t; };
template<typename T> concept C2 = C1<T> && requires(T t) { *t; };

template<C1 T> void f(T);       // \#1
template<C2 T> void f(T);       // \#2
template<typename T> void g(T); // \#3
template<C1 T> void g(T);       // \#4

f(0);       // selects \#1
f((int*)0); // selects \#2
g(true);    // selects \#3 because \tcode{C1<bool>} is not satisfied
g(0);       // selects \#4
\end{codeblock}
\exitexample

\end{addedblock}
\end{quote}
