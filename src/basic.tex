%!TEX root = ts.tex
\rSec0[basic]{Basic concepts}

\rSec1[gram.basic]{Basic concepts}

Add concepts to the list of entities.

\begin{quote}
\setcounter{Paras}{2}
\pnum
An \defn{entity} is a value, object, reference, function, enumerator, type,
class member, bit-field, template, \added{concept,} template specialization, 
namespace, or parameter pack.
\end{quote}

\rSec1[basic.def.odr]{One-definition rule}

Modify at the beginning of paragraph 6.

\begin{quote}
\setcounter{Paras}{5}
\pnum
There can be more than one definition of a class type
(Clause~\cxxref{class}), enumeration type~(\cxxref{dcl.enum}), inline function
with external linkage~(\cxxref{dcl.inline}), inline variable with external
linkage~(\cxxref{dcl.inline}), class template
(Clause~\ref{temp}), non-static function template~(\ref{temp.fct}),
static data member of a class template~(\cxxref{temp.static}), member
function of a class template~(\ref{temp.mem.func}), \removed{or} template
specialization for which some template parameters are not
specified~(\ref{temp.spec}, \ref{temp.class.spec})\added{,}
\added{or concept} in a program provided
that each definition appears in a different translation unit, and
provided the definitions satisfy the following requirements.
\end{quote}
