
%%
%% Templates
%%
\setcounter{chapter}{16}
\rSec0[temp]{Templates}

Modify the \grammarterm{template-declaration} grammar in paragraph 1.

\begin{quote}
\pnum
A \defn{template} defines a family of classes, functions, or variables, or an alias for a
family of types.

\begin{bnf}
\nontermdef{template-declaration}\br
  \added{template-head} \removed{\terminal{template} \terminal{<} template-parameter-list \terminal{>}} declaration\br

\begin{addedblock}
\nontermdef{template-head}\br
  \terminal{template} \terminal{<} template-parameter-list \terminal{>} requires-clause\opt\br
\end{addedblock}

\begin{addedblock}
\nontermdef{requires-clause}\br
  \terminal{requires} constraint-expression
\end{addedblock}
\end{bnf}

\end{quote}
  
Add the following paragraphs after paragraph 6.

\begin{quote}
\begin{addedblock}
\setcounter{Paras}{6}
\pnum
A \grammarterm{template-declaration} is written in terms of its template 
parameters. 
%
The optional \grammarterm{requires-clause} following a
\grammarterm{template-parameter-list} allows the specification of
constraints (\ref{temp.constr.decl}) on template arguments (\ref{temp.arg}).
\end{addedblock}
\end{quote}


%%
%% Template parameters
%%
\rSec1[temp.param]{Template parameters}

In paragraph 1, extend the grammar for template parameters to 
constrained template parameters.

\begin{quote}
\pnum
The syntax for \grammarterm{template-parameter}{s} is:

\begin{bnf}
\nontermdef{template-parameter}\br
  type-parameter\br
  parameter-declaration\br
  \added{constrained-parameter}
\end{bnf}

\begin{bnf}
\nontermdef{type-parameter}\br
  type-parameter-key \terminal{...}\opt identifier\opt\br
  type-parameter-key identifier\opt{} \terminal{=} type-id\br
  \terminal{template <} template-parameter-list \terminal{>} type-parameter-key \terminal{...}\opt identifier\opt\br
  \terminal{template <} template-parameter-list \terminal{>} type-parameter-key identifier\opt{} \terminal{=} id-expression
\end{bnf}

\begin{bnf}
\nontermdef{type-parameter-key}\br
  \terminal{class}\br
  \terminal{typename}
\end{bnf}

\begin{bnf}
\begin{addedblock}
\nontermdef{constrained-parameter}\br
  qualified-concept-name ... identifier\opt\br
  qualified-concept-name identifier\opt default-template-argument\opt

\nontermdef{qualified-concept-name}\br
	nested-name-specifier\opt concept-name\br
	nested-name-specifier\opt partial-concept-id

\nontermdef{partial-concept-id}\br
		concept-name \terminal{<} template-argument-list\opt \terminal{>}

\nontermdef{default-template-argument}\br
  \terminal{=} type-id\br
  \terminal{=} id-expression\br
  \terminal{=} initializer-clause
\end{addedblock}
\end{bnf}
\end{quote}

Insert the following paragraphs after paragraph 8 in the \Cpp Standard. These 
paragraphs define the meaning of a constrained template parameter.

\begin{quote}
\begin{addedblock}
\setcounter{Paras}{8}
\pnum
A \grammarterm{partial-concept-id} is a \grammarterm{concept-name} followed 
by a sequence of \grammarterm{template-argument}{}s. These template arguments
are used to form a \grammarterm{constraint-expression} as described below.

\pnum
A \grammarterm{constrained-parameter} declares a template parameter whose 
kind (type, non-type, template) and type match that of the prototype parameter 
(\ref{temp.concept}) of the concept designated by the 
\grammarterm{qualified-concept-name} in the \grammarterm{constrained-parameter}.
%
Let \tcode{X} be the prototype parameter of the designated concept.
% 
The declared template parameter is determined by the kind of \tcode{X} 
(type, non-type, template) and the optional ellipsis in the
\grammarterm{constrained-parameter} as follows.
% 
\begin{itemize}
\item If \tcode{X} is a type \grammarterm{template-parameter}, the declared
parameter is a type \grammarterm{template-parameter}. 

\item If \tcode{X} is a non-type \grammarterm{template-parameter}, the declared
parameter is a non-type \grammarterm{template-parameter} having the same 
type as \tcode{X}.

\item If \tcode{X} is a template \grammarterm{template-parameter}, the declared
parameter is a template \grammarterm{template-parameter} having the same 
\grammarterm{template-parameter-list} as \tcode{X}, excluding default template 
arguments.

\item If the \grammarterm{qualified-concept-name} is followed by an ellipsis,
then the declared parameter is a template parameter pack (\cxxref{temp.variadic}).
\end{itemize}
% 
\enterexample
\begin{codeblock}
template<typename T> concept C1 = true;
template<template<typename> class X> concept C2 = true;
template<int N> concept C3 = true;
template<typename... Ts> concept C4 = true;
template<char... Cs> concept C5 = true;

template<C1 T> void f1();     // OK: \tcode{T} is a type template-parameter
template<C2 X> void f2();     // OK: \tcode{X} is a template with one type-parameter
template<C3 N> void f3();     // OK: \tcode{N} has type \tcode{int}
template<C4... Ts> void f4(); // OK: \tcode{Ts} is a template parameter pack of types
template<C4 T> void f5();     // OK: \tcode{T} is a type template-parameter
template<C5... Cs> void f6(); // OK: \tcode{Cs} is a template parameter pack of \tcode{char}s
\end{codeblock}
\exitexample

\pnum
A \grammarterm{constrained-parameter} introduces a 
\grammarterm{constraint-expression} (\ref{temp.constr.decl}).
% 
The expression is derived from the \grammarterm{qualified-concept-name} 
\tcode{Q} in the \grammarterm{constrained-parameter}, its designated concept 
\tcode{C}, and the declared template parameter \tcode{P}.
% 
\begin{itemize}
\item First, form a template argument \tcode{A} from \tcode{P}. If \tcode{P} 
declares a template parameter pack (\cxxref{temp.variadic})
and \tcode{C} is a variadic concept (\ref{temp.concept}), then \tcode{A} is 
the pack expansion \tcode{P...}. Otherwise, \tcode{A} is the 
\grammarterm{id-expression} \tcode{P}.

% FIXME: This does not guarantee that the expression has the same 
% namespace qualification as Q.
\item Then, form an \grammarterm{id-expression} \tcode{E} as follows.
If \tcode{Q} is a \grammarterm{concept-name}, then \tcode{E} is \tcode{C<A>}. 
Otherwise, \tcode{Q} is a \grammarterm{partial-concept-id} of the form 
\tcode{C<A1, A2, ..., A$N$>}, and \tcode{E} is \tcode{C<A, A1, A2, ..., A$N$>}.

\item Finally, if \tcode{P} declares a template parameter pack and 
\tcode{C} is not a variadic concept, \tcode{E} is adjusted to be the
\grammarterm{fold-expression} \tcode{(E \&\& ...)} (\cxxref{expr.prim.fold}).
\end{itemize}
% 
\tcode{E} is the introduced \grammarterm{constraint-expression}.
% 
\enterexample
\begin{codeblock}
template<typename T> concept C1 = true;
template<typename... Ts> concept C2 = true;
template<typename T, typename U> concept C3 = true;

template<C1 T> struct s1;      // associates \tcode{C1<T>}
template<C1... T> struct s2;   // associates \tcode{(C1<T> \&\& ...)}
template<C2... T> struct s3;   // associates \tcode{C2<T...>}
template<C3<int> T> struct s4; // associates \tcode{C3<T, int>}
\end{codeblock}
\exitexample
\end{addedblock}
\end{quote}

Insert the following paragraph after paragraph 9 in the \Cpp Standard to 
require that the kind of a \grammarterm{default-argument} matches the kind 
of its \grammarterm{constrained-parameter}.

\begin{quote}
\begin{addedblock}
\setcounter{Paras}{11}
\pnum
The default \grammarterm{template-argument} of
a \grammarterm{constrained-parameter} shall match
the kind (type, non-type, template) of the declared template parameter.
% 
\enterexample
\begin{codeblock}
template<typename T> concept C1 = true;
template<int N> concept C2 = true;
template<template<typename> class X> concept C3 = true;

template<typename T> struct S0;

template<C1 T = int> struct S1; // OK
template<C2 N = 0> struct S2;   // OK
template<C3 X = S0> struct S3;  // OK
template<C1 T = 0> struct S4;   // error: default argument is not a type
\end{codeblock}
\exitexample
\end{addedblock}
\end{quote}


%%
%% Names of template specializations
%%
\rSec1[temp.names]{Names of template specializations}

Add this paragraph at the end of the section to require the satisfaction of 
associated constraints on the formation of the \grammarterm{simple-template-id}.

\begin{quote}
\begin{addedblock}
\setcounter{Paras}{7}
\pnum
When the \grammarterm{template-name} of a \grammarterm{simple-template-id} names
a constrained non-function template or a constrained template 
\grammarterm{template-parameter}, but not a member template that is a member 
of an unknown specialization (\ref{temp.res}), and all 
\grammarterm{template-argument}{s} in the \grammarterm{simple-template-id} 
are non-dependent (\cxxref{temp.dep.temp}), the associated constraints of the 
constrained template shall be satisfied. (\ref{temp.constr.decl}).
% 
\enterexample
\begin{codeblock}
template<typename T> concept C1 = sizeof(T) != sizeof(int);

template<C1 T> struct S1 { };
template<C1 T> using Ptr = T*;

S1<int>* p; // error: constraints not satisfied
Ptr<int> p; // error: constraints not satisfied

template<typename T>
  struct S2 { Ptr<int> x; }; // error, no diagnostic required

template<typename T>
  struct S3 { Ptr<T> x; };   // OK: satisfaction is not required

S3<int> x;                   // error: constraints not satisfied

template<template<C1 T> class X>
  struct S4 {
    X<int> x; // error, no diagnostic required
  };

template<typename T> concept C2 = sizeof(T) == 1;

template<C2 T> struct S { };

template struct S<char[2]>;       // error: constraints not satisfied
template<> struct S<char[2]> { }; // error: constraints not satisfied
\end{codeblock}
\end{addedblock}
\end{quote}

%%
%% Template arguments
%%
\rSec1[temp.arg]{Template arguments}

%%
%% Template template arguments
%%
\setcounter{subsection}{2}
\rSec2[temp.arg.template]{Template template arguments}


Add the following example to the end of paragraph 3, after the
examples given in the \Cpp Standard.

\begin{quote}
\begin{addedblock}
\enterexample
\begin{codeblock}
template<typename T> concept C = requires (T t) { t.f(); };
template<typename T> concept D = C<T> && requires (T t) { t.g(); };

template<template<C> class P>
  struct S { };

template<C> struct X { };
template<D> struct Y { };
template<typename T> struct Z { };

S<X> s1; // OK: \tcode{X} and \tcode{P} have equivalent constraints
S<Y> s2; // error: \tcode{P} is not at least as specialized as \tcode{Y}
S<Z> s3; // OK: \tcode{P} is at least as specialized as \tcode{Z}
\end{codeblock}
\exitexample
\end{addedblock}
\end{quote}


%%
%% Template declarations
%%
\setcounter{section}{4}
\rSec1[temp.decls]{Template declarations}

Modify paragraph 2 to indicate that associated constraints are
instantiated separately from the template they are associated with.

\begin{quote}
\setcounter{Paras}{1}
\pnum
For purposes of name lookup and instantiation,
default arguments\added{, \grammarterm{requires-clause}{s} (Clause~\ref{temp}),} 
and \grammarterm{noexcept-specifier}{s} of function templates and 
default arguments\added{, \grammarterm{requires-clause}{s},} and 
\grammarterm{noexcept-specifier}{s} of member functions of class templates 
are considered definitions;
each default argument\added{, \grammarterm{requires-clause},} or 
\grammarterm{noexcept-specifier} is a separate definition which is unrelated to
the function template definition or to any other default arguments or
\grammarterm{noexcept-specifier}{s}.
For the purpose of instantiation, the substatements of a constexpr if
statement~(\cxxref{stmt.if}) are considered definitions.
\end{quote}


%%
%% Class templates
%%
\rSec2[temp.class]{Class templates}

Modify paragraph 3 to require template constraints for out-of-class
definitions of members of constrained templates. 

\begin{quote}
\setcounter{Paras}{2}
\pnum
When a member function, a member class, a member enumeration, a static 
data member or a member template of a class template is defined outside 
of the class template definition, the member definition is defined as a 
template definition in which the
\removed{\grammarterm{template-parameter}{s} are those}
\added{\grammarterm{template-head} is equivalent to that}
of the class template \added{(\ref{temp.over.link})}.
% 
The names of the template parameters used in the definition of the 
member may be different from the template parameter names used in the 
class template definition. The template argument list following the class
template name in the member definition shall name the parameters in the 
same order as the one used in the template parameter list of the member. 
% 
Each template parameter pack shall be expanded with an ellipsis in the 
template argument list.
\end{quote}

Add the following example at the end of paragraph 3.

\begin{quote}
\begin{addedblock}
\enterexample
\begin{codeblock}
template<typename T> concept C = true;
template<typename T> concept D = true;

template<C T> struct S {
  void f();
  void g();
  void h();
  template<D U> struct Inner;
};

template<C A> void S<A>::f() { }        // OK: \grammarterm{template-head}{s} match
template<typename T> void S<T>::g() { } // error: no matching declaration for \tcode{S<T>}

template<typename T> requires C<T>      // error (no diagnostic required): \grammarterm{template-head}{s} are
void S<T>::h() { }                      // functionally equivalent but not equivalent

template<C X> template<D Y> struct S<X>::Inner { }; // OK
\end{codeblock}
\exitexample
\end{addedblock}
\end{quote}


%%
%% Member functions of class templates
%%
\rSec3[temp.mem.func]{Member functions of class templates}

Add the following example to the end of paragraph 1.

\begin{quote}
\begin{addedblock}
\enterexample
\begin{codeblock}
template<typename T> concept C = requires {
  typename T::type;
};

template<typename T> struct S {
  void f() requires C<T>;
  void g() requires C<T>;
};

template<typename T> 
  void S<T>::f() requires C<T> { } // OK
template<typename T> 
  void S<T>::g() { }               // error: no matching function in \tcode{S<T>}
\end{codeblock}
\exitexample
\end{addedblock}
\end{quote}


%%
%% Member templates
%%
\rSec2[temp.mem]{Member templates}


Modify paragraph 1 in order to account for constrained member templates
of (possibly) constrained class templates. 

\begin{quote}
\pnum
A template can be declared within a class or class template; such a 
template is called a member template. 
% 
A member template can be defined within or outside its class definition 
or class template definition. 
% 
A member template of a class template that is defined outside of its 
class template definition shall be specified with
\removed{the \grammarterm{template-parameter}{s}}
\added{a \grammarterm{template-head} equivalent to that}
of the class template followed by
\removed{the \grammarterm{template-parameter}{s}}
\added{a \grammarterm{template-head} equivalent to that}
of the member template.
\end{quote}


Add the following example at the end of paragraph 1.

\begin{quote}
\begin{addedblock}
\enterexample
\begin{codeblock}
template<typename T> concept C1 = true;
template<typename T> concept C2 = sizeof(T) <= 4;

template<C1 T>
  struct S {
    template<C2 U> void f(U);
    template<C2 U> void g(U);
  };

template<C1 T> template<C2 U> 
  void S<T>::f(U) { } // OK
template<C1 T> template<typename U> 
  void S<T>::g(U) { } // error: no matching function in \tcode{S<T>}
\end{codeblock}
\exitexample
\end{addedblock}
\end{quote}


%%
%% Friends
%%
\rSec2[temp.friend]{Friends}

Add a new paragraph to make non-template friends ill-formed.

\begin{quote}
\begin{addedblock}
\setcounter{Paras}{8}
\pnum
A non-template friend declaration shall not have a \grammarterm{requires-clause}.
\end{addedblock}
\end{quote}

% When a friend declaration refers to a specialization of a function
% template, the function parameter declarations shall not include
% default arguments, \added{the declaration shall not have associated constraints
% (\ref{temp.constr.decl}),} nor shall the inline specifier be used in such a
% declaration.
% \end{quote}

% Add examples following that paragraph.

% \begin{quote}
% \begin{addedblock}
% \pnum
% \enternote
% Other friend declarations can be constrained. 
% \exitnote
% \enterexample
% \begin{codeblock}
% template<typename T> concept C1 = true;
% template<typename T> concept C2 = false;

% template<C1 U> void g0(U);
% template<C1 U> void g1(U);
% template<C2 U> void g2(U);

% template<typename T>
%   struct S {
%     friend void f1() requires C1<void>;  // OK
%     friend void f2() requires C1<T>;     // OK
%     friend void g0<T>(T) requires C1<T>; // error: constrained friend specialization
%     friend void g1<T>(T);                // OK
%     friend void g2<T>(T);                // error: constraint can never be satisfied, 
%                                          // no diagnostic required
%   };

% void f1() requires C1<void>;   // friend of any \tcode{S<T>}
% void f2() requires C1<int>;    // friend of only \tcode{S<int>}
% \end{codeblock}
% The friend declaration of \tcode{g2} is ill-formed, no
% diagnostic required, because no valid specialization of \tcode{S}
% can be generated: the constraint on \tcode{g2} can never
% be satisfied, so template argument deduction
% (\ref{temp.deduct.decl}) will always fail.
% % 
% If there are overloads of e.g., \tcode{g1}, the most specialized
% is selected as the friend declaration when \tcode{S} is instantiated.
% \exitexample

% \pnum
% \enternote
% Within a class template, a friend may define a non-template function
% whose constraints specify requirements on template arguments.
% \enterexample
% \begin{codeblock}
% template<typename T> concept Eq = requires (T t) { t.equal(t); };

% template<typename T>
%   struct S {
%     friend bool operator==(S a, S b) requires Eq<T> { return a.equal(b); } // OK
%   };
% \end{codeblock}
% \exitexample
% Constraints are checked (\ref{temp.constr}) only when
% that function is considered as a viable candidate for overload resolution
% (\ref{over.match.viable}). If substitution fails, the program is ill-formed.
% \exitnote
% \end{addedblock}
% \end{quote}


%%
%% Class template partial specializations
%%
\rSec2[temp.class.spec]{Class template partial specialization}

After paragraph 3, insert the following, which allows constrained partial 
specializations.

\begin{quote}
\begin{addedblock}
\setcounter{Paras}{3}
\pnum
A class template partial specialization may be constrained
(Clause~\ref{temp}).
\enterexample
\begin{codeblock}
template<typename T> concept C = requires (T t) { t.f(); };
template<int I> concept N = I > 0;

template<C T1, C T2, N I> class A<T1, T2, I>;  // \#6
template<C T, N I>        class A<int, T*, I>; // \#7
\end{codeblock}
\exitexample
\end{addedblock}
\end{quote}

%%
%% Matching of class template partial specializations
%%
\rSec3[temp.class.spec.match]{Matching of class template partial specializations}

Modify paragraph 2; constraints must be satisfied in order
to match a partial specialization. 

\begin{quote}
\setcounter{Paras}{1}
\pnum
A partial specialization matches a given actual template argument list if 
the template arguments of the partial specialization can be deduced from the 
actual template argument list (\ref{temp.deduct})\added{, and the deduced 
template arguments satisfy the constraints of the partial specialization, if 
any (\ref{temp.constr.decl})}.
\end{quote}

Add the following example to the end of paragraph 2.

\begin{quote}
\begin{addedblock}
\enterexample
\begin{codeblock}
struct S { void f(); };

A<S, S, 1>    a6; // uses \#6
A<int, S*, 3> a8; // uses \#7
\end{codeblock}
\exitexample
\end{addedblock}
\end{quote}


%%
%% Partial ordering of class template specializations
%%
\rSec3[temp.class.order]{Partial ordering of class template specializations}

Modify paragraph 1 so that constraints are considered in the
partial ordering of class template specializations. 

\begin{quote}
\pnum
For two class template partial specializations, the first is 
at least as specialized as the second if, given the following 
rewrite to two function templates, the first function template 
is at least as specialized as the second according to the ordering 
rules for function templates 
(\ref{temp.func.order}):
% 
\begin{itemize}
\item the first function template has the same template 
parameters \added{and associated constraints (\ref{temp.constr.decl})} 
as the first partial specialization, and has a single function parameter 
whose type is a class template specialization with the template
arguments of the first partial specialization, and

\item the second function template has the same template 
parameters \added{and associated constraints (\ref{temp.constr.decl})} 
as the second partial specialization, and has a single function parameter 
whose type is a class template specialization with the template
arguments of the second partial specialization.
\end{itemize}
\end{quote}

Add the following example to the end of paragraph 1.

\begin{quote}
\begin{addedblock}
\enterexample
\begin{codeblock}
template<typename T> concept C = requires (T t) { t.f(); };
template<typename T> concept D = C<T> && requires (T t) { t.f(); };


template<typename T> class S { };
template<C T> class S<T> { }; // \#1
template<D T> class S<T> { }; // \#2

template<C T> void f(S<T>); // A
template<D T> void f(S<T>); // B
\end{codeblock}
The partial specialization \#2 is more specialized than 
\#1 because \tcode{B} is more specialized than \tcode{A}.
\exitexample
\end{addedblock}
\end{quote}


%%
%% Function templates
%%
\rSec2[temp.fct]{Function templates}

%%
%% Function template overloading
%%
\rSec3[temp.over.link]{Function template overloading}

Modify paragraph 6 to expand the scope of rules governing the existing term
"equivalent".

\begin{quote}
\setcounter{Paras}{5}
\pnum
Two function templates are \defn{equivalent} if they are 
declared in the same scope, have the same name,
\removed{have identical template parameter lists}
and have \added{\grammarterm{template-head}{s},}
return types\added{,} and parameter lists that are 
equivalent using the rules described above to compare expressions 
involving template parameters.
% 
Two function templates are \defn{functionally equivalent} if they 
are equivalent except that one or more expressions that involve 
template parameters in the \added{\grammarterm{template-head}{s},}
return types\added{,} and parameter lists
are functionally equivalent using the rules described above to compare 
expressions involving template parameters.
% 
If \removed{a program contains declarations of function templates that}
\added{the validity or meaning of the program depends on whether two
constructs are equivalent, and they}
are 
functionally equivalent but not equivalent, the program is ill-formed; 
no diagnostic is required.
\end{quote}


%%
%% Partial ordering of function templates
%%
\rSec3[temp.func.order]{Partial ordering of function templates}

Modify paragraph 2 to include constraints in the partial ordering
of function templates.

\begin{quote}
\setcounter{Paras}{1}
\pnum
Partial ordering selects which of two function templates is 
more specialized than the other by transforming each template 
in turn (see next paragraph) and performing template argument 
deduction using the function type. The deduction process 
determines whether one of the templates is more specialized 
than the other.
% 
If so, the more specialized template is the one chosen by the 
partial ordering process. 
% 
\added{If both deductions succeed, the partial ordering selects
the more constrained template as described by the rules in
\ref{temp.constr.order}.}
\end{quote}

%%
%% Concept definitions
%%
\rSec2[temp.concept]{Concept definitions}

Add the following paragraph.

\begin{quote}
\begin{addedblock}
\pnum A \defn{concept} defines constraints on its template arguments.

\begin{bnf}
\nontermdef{concept-definition}\br
  \terminal{concept} concept-name \terminal{=} constraint-expression\br
\end{bnf}
\begin{bnf}
\nontermdef{concept-name}\br
  identifier
\end{bnf}

\pnum A \grammarterm{concept-definition} declares its \grammarterm{identifier}
to be a concept. The name of the concept is a \grammarterm{concept-name}.
\enterexample
\begin{codeblock}
template<typename T>
concept C = requires(T x) {
  { x == x } -> bool;
};
template<typename T> 
  requires C<T> // \tcode{C} constrains \tcode{f1(T)} in \grammarterm{constraint-expression}
T f1(T x) { return x; }
template<C T> // \tcode{C} constrains \tcode{f2(T)} as a \grammarterm{constrained-parameter}
T f2(T x) { return x; }
\end{codeblock}
\exitexample

\pnum
A \grammarterm{concept-definition} shall appear in the global scope or in a
namespace scope \cxxref{basic.scope.namespace}.

\pnum
A concept is not instantiated (\ref{temp.spec}). 
A program that explicitly instantiates (\ref{temp.explicit}), explicitly 
specializes (\ref{temp.expl.spec}), or partially
specializes a concept is ill-formed. 
\enternote
\emph{id-expression}s that denote a concept specialization are evaluated as
expressions (\ref{expr.prim.id}).
\exitnote

\pnum
The first declared template parameter of a concept definition is its 
\defn{prototype parameter}.
A \defn{variadic concept} is a concept whose prototype parameter is a template 
parameter pack.
\end{addedblock}
\end{quote}

%%
%% Name resolution
%%
\rSec1[temp.res]{Name resolution}

Modify paragraph 8.

\begin{quote}
\setcounter{Paras}{7}
\pnum
Knowing which names are type names allows the syntax of every
template to be checked. No diagnostic shall be issued for a template
for which a valid specialization can be generated. If no valid
specialization can be generated for a template, and that template is
not instantiated, the template is ill-formed, no diagnostic
required. If every valid specialization of a variadic template
requires an empty template parameter pack, the template is
ill-formed, no diagnostic required. 
% 
% TODO: "would result in a valid expression" should probably be
% "would always result in substitution failure".
% 
\added{If no instantiation of
the associated constraints (\ref{temp.constr.decl}) of a template would result 
in a valid expression, the template is ill-formed, no diagnostic required.}
% 
If a hypothetical instantiation of a template immediately following
its definition would be ill-formed due to a construct that does not
depend on a template parameter, the program is ill-formed; no
diagnostic is required. If the interpretation of such a construct in
the hypothetical instantiation is different from the interpretation
of the corresponding construct in any actual instantiation of the
template, the program is ill-formed; no diagnostic is required.
\end{quote}

%%
%% Dependent name resolution
%% 
\setcounter{subsection}{3}
\rSec2[temp.dep.res]{Dependent name resolution}

%%
%% Point of instantiation
%%
\rSec3[temp.point]{Point of instantiation}

Add a new paragraph after paragraph 4.

\begin{quote}
\setcounter{Paras}{4}
\begin{addedblock}
\pnum
The point of instantiation of a \grammarterm{constraint-expression} of a
specialization immediately precedes the point of instantiation of
the specialization.
\end{addedblock}
\end{quote}


%%
%% Template instantiation and specialization
%%
\rSec1[temp.spec]{Template instantiation and specialization}

%%
%% Implicit instantiation
\rSec2[temp.inst]{Implicit Instantiation}
    
Change paragraph 1 to include associated constraints.

\begin{quote}
\pnum
Unless a class template specialization has been explicitly
instantiated~(\ref{temp.explicit}) or explicitly
specialized~(\ref{temp.expl.spec}),
the class template specialization is implicitly instantiated when the
specialization is referenced in a context that requires a completely-defined
object type or when the completeness of the class type affects the semantics
of the program.
\begin{note}
In particular, if the semantics of an expression depend on the member or
base class lists of a class template specialization, the class template
specialization is implicitly generated. For instance, deleting a pointer
to class type depends on whether or not the class declares a destructor,
and a conversion between pointers to class type depends on the
inheritance relationship between the two classes involved.
\end{note}
\enterexample
\begin{codeblock}
template<class T> class B { @. . .@ };
template<class T> class D : public B<T> { @. . .@ };

void f(void*);
void f(B<int>*);

void g(D<int>* p, D<char>* pp, D<double>* ppp) {
  f(p);             // instantiation of \tcode{D<int>} required: call \tcode{f(B<int>*)}
  B<char>* q = pp;  // instantiation of \tcode{D<char>} required: convert \tcode{D<char>*} to \tcode{B<char>*}
  delete ppp;       // instantiation of \tcode{D<double>} required
}
\end{codeblock}
\exitexample
If a class template has been declared, but not defined,
at the point of instantiation~(\ref{temp.point}),
the instantiation yields an incomplete class type~(\ref{basic.types}).
\enterexample
\begin{codeblock}
template<class T> class X;
X<char> ch;         // error: incomplete type \tcode{X<char>}
\end{codeblock}
\exitexample
\begin{note}
Within a template declaration,
a local class~(\ref{class.local}) or enumeration and the members of
a local class are never considered to be entities that can be separately
instantiated (this includes their default arguments, 
\added{associated constraints (\ref{temp.constr.decl}),}
\grammarterm{noexcept-specifier}{s}, and non-static data member
initializers, if any). As a result, the dependent names are looked up, the
semantic constraints are checked, and any templates used are instantiated as
part of the instantiation of the entity within which the local class or
enumeration is declared.
\end{note}
\end{quote}


Add a new paragraph at the end of this section to describe how associated
constraints are instantiated.

\begin{quote}
\begin{addedblock}
\setcounter{Paras}{15}
\pnum
The associated constraints of a template specialization are not
instantiated along with the specialization itself; they are
instantiated only to determine if they are satisfied
(\ref{temp.constr.decl}).
% 
\enternote
The satisfaction of constraints is determined during name lookup or overload
resolution (\ref{over.match}).
\exitnote
% 
\enterexample
\begin{codeblock}
template<typename T> concept C = sizeof(T) > 2;
template<typename T> concept D = C<T> && sizeof(T) > 4;

template<typename T> struct S {
  S() requires C<T> { } // \#1
  S() requires D<T> { } // \#2
};

S<char> s1;    // error: no matching constructor
S<char[8]> s2; // OK: calls \#2
\end{codeblock}

Even though neither constructor for \tcode{S<char>} will be selected by
overload resolution, both constructors remain a part of the class template 
specialization. 
% 
This also has the effect of suppressing the implicit generation of a default
constructor (\cxxref{class.ctor}).
\exitexample
% 
\enterexample
\begin{codeblock}
template<typename T> struct S1 {
  template<typename U> requires false struct Inner1; // OK
};

template<typename T> struct S2 {
  template<typename U> 
    requires sizeof(T[-(int)sizeof(T)]) > 1 // error: ill-formed, no diagnostic required
      struct Inner2;
};
\end{codeblock}
\exitexample
Every instantiation of \tcode{S1} results in a valid type, although any use 
of its nested \tcode{Inner1} template is invalid.
% 
\tcode{S2} is ill-formed, no diagnostic required, since no substitution into 
the constraints of its \tcode{Inner2} template would result in a valid 
expression.
\end{addedblock}
\end{quote}


%%
%% Explicit instantiation
%%
\rSec2[temp.explicit]{Explicit instantiation}

Add the following note after paragraph 7.

\begin{quote}
\setcounter{Paras}{7}
\begin{addedblock}
\pnum
\enternote
An explicit instantiation of a constrained template shall satisfy that
template's associated constraints (\ref{temp.constr.decl}). The satisfaction
of constraints is determined during name lookup for explicit instantiations
in which all template arguments are specified (\ref{temp.names}), or for
explicit instantiations of function templates, during template argument 
deduction (\ref{temp.deduct.decl}) when one or more trailing template 
arguments are left unspecified.
\exitnote
\end{addedblock}
\end{quote}

Modify paragraph 8 in the \Cpp standard (paragraph 9, here) to ensure that 
only members whose constraints are satisfied are explicitly instantiated 
during class template specialization. The note in the \Cpp Standard is omitted.

\begin{quote}
\setcounter{Paras}{8}
\pnum
An explicit instantiation that names a class template specialization is 
also an explicit instantiation of the same kind (declaration or 
definition) of each of its members (not including members inherited from 
base classes and members that are templates) that has not been previously 
explicitly specialized in the translation unit containing the explicit 
instantiation, \added{and provided that the associated constraints, if any, 
of that member are satisfied by the template arguments of the explicit 
instantiation (\ref{temp.constr.decl}),}
except as described below.
\end{quote}

%%
%% Explicit specialization
%%
\rSec2[temp.expl.spec]{Explicit specialization}

Add the following note after paragraph 10.

\begin{quote}
\setcounter{Paras}{10}
\begin{addedblock}
\pnum
\enternote
An explicit specialization of a constrained template shall satisfy that
template's associated constraints (\ref{temp.constr.decl}). The satisfaction
of constraints is determined during name lookup for explicit specializations
in which all template arguments are specified (\ref{temp.names}), or for
explicit specializations of function templates, during template argument 
deduction (\ref{temp.deduct.decl}) when one or more trailing template 
arguments are left unspecified.
\exitnote
\end{addedblock}
\end{quote}


%%
%% Function template specializations
%%
\rSec1[temp.fct.spec]{Function template specializations}

%%
%% Template argument deduction
%%
\setcounter{subsection}{1}
\rSec2[temp.deduct]{Template argument deduction}

Add the following sentences to the end of paragraph 5. This defines
the substitution of template arguments into a function template's
associated constraints. Note that the last part of paragraph 5
has been duplicated in order to provide context for the addition.

\begin{quote}
\setcounter{Paras}{4}
\pnum
When all template arguments have been deduced or obtained from default 
template arguments, all uses of template parameters in the template 
parameter list of the template and the function type are replaced with
the corresponding deduced or default argument values. 
% 
If the substitution results in an invalid type, as described above, type 
deduction fails.
% 
\added{If the function template has associated constraints (\ref{temp.constr.decl}),
the template arguments are substituted into the associated constraints
without evaluating the resulting expression. If this substitution
results in an invalid type or expression, type deduction fails.
% 
\enternote
The satisfaction of constraints (\ref{temp.constr})
associated with the function template specialization is determined during 
overload resolution (\ref{over.match}), and not at 
the point of substitution.
\exitnote}
\end{quote}


\setcounter{subsubsection}{5}
\rSec3[temp.deduct.decl]{Deducing template arguments from a function declaration}

Add a new paragraph after paragraph 1 in order to require the
satisfaction of constraints when matching a specialization to a
template.

\begin{quote}
\begin{addedblock}
\setcounter{Paras}{1}
\pnum
Remove from the set of function templates considered all those
whose associated constraints (if any) are not satisfied by the deduced
template arguments (\ref{temp.constr.decl}).
\end{addedblock}
\end{quote}

Update paragraph 2 in the \Cpp Standard (paragraph 3, here) to accommodate 
the new wording.

\begin{quote}
\pnum
If, 
\removed{for the set of function templates so considered}
\added{for the remaining function templates},
there is either no match or more than one match after partial ordering 
has been considered (\ref{temp.func.order}), deduction fails 
and, in the declaration cases, the program is ill-formed.
\end{quote}

Add the following example to paragraph 3.

\begin{quote}
\enterexample
\begin{codeblock}
template<typename T> concept C = requires (T t) { -t; };

template<C T>        void f(T) { } // \#1
template<typename T> void g(T) { } // \#2
template<C T>        void g(T) { } // \#3

template void f(int);   // OK: refers to \#1
template void f(void*); // error: no matching template
template void g(int);   // OK: refers to \#3
template void g(void*); // OK: refers to \#2
\end{codeblock}
\exitexample
\end{quote}
