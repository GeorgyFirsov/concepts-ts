%%
%% Terms and definitions
%%
\rSec0[intro.defs]{Terms and definitions}

Modify the definitions of ``signature'' to include constraints. This allows 
different translation units to contain definitions of functions with the same 
signature, excluding constraints, without violating the one definition rule 
(\cxxref{basic.def.odr}). That is, without incorporating the constraints
in the signature, such functions would have the same mangled name, thus
appearing as multiple definitions of the same function.

\begin{quote}
\indexdefn{signature}%
\definition{signature}{defns.signature}
<function> name, parameter type list~(\ref{dcl.fct}), \removed{and} enclosing 
namespace (if any)%
\added{, and \grammarterm{requires-clause} (\ref{temp.constr.decl}) (if any)}\\
\enternote Signatures are used as a basis for
name mangling and linking.\exitnote

\indexdefn{signature}%
\definition{signature}{defns.signature.templ}
<function template> name, parameter type list~(\ref{dcl.fct}), enclosing namespace (if any),
return type,
\removed{and template parameter list}
\added{\grammarterm{template-head}, and \grammarterm{requires-clause}~(\ref{temp.constr.decl}) (if any)}

\indexdefn{signature}%
\definition{signature}{defns.signature.member}
<class member function> name, parameter type list~(\ref{dcl.fct}), class of which the
function is a member, \cv-qualifiers (if any),
\removed{and} \grammarterm{ref-qualifier} (if any)%
\added{, and \grammarterm{requires-clause}~(\ref{temp.constr.decl}) (if any)}\\

\indexdefn{signature}%
\definition{signature}{defns.signature.member.templ}
<class member function template> name, parameter type list~(\ref{dcl.fct}), class of which the
function is a member, \cv-qualifiers (if any),
\grammarterm{ref-qualifier} (if any), return type,
\removed{and template parameter list}
\added{\grammarterm{template-head}, and \grammarterm{requires-clause}~(\ref{temp.constr.decl}) (if any)}
\end{quote}
